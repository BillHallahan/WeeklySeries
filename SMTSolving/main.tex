\documentclass{article}%{llncs2e/llncs}
%\documentclass[nocopyrightspace,10pt,twocolumn]{sigplanconf}
%\documentclass[10pt,twocolumn]{sigplanconf}

% The following \documentclass options may be useful:

% preprint      Remove this option only once the paper is in final form.
% 10pt          To set in 10-point type instead of 9-point.
% 11pt          To set in 11-point type instead of 9-point.
% authoryear    To obtain author/year citation style instead of numeric.



\usepackage{amsmath}
\usepackage[USenglish,english,american]{babel}
\usepackage{times, color}
\usepackage{comment, cite, xspace, graphicx, subfig, footnote}
\usepackage[algoruled,vlined,linesnumbered,ruled]{algorithm2e}
%\usepackage[compact]{titlesec}
\usepackage{microtype}
\usepackage{listings}
\usepackage{float}
\usepackage{hyperref}
\lstset{ %
  language=C,
  basicstyle=\footnotesize\ttfamily        % the size of the fonts that are used for the code
}



\pagestyle{plain}

%%%%%%%%%%%%%%%%%%%%%%%%%%%%

\begin{document}

\special{papersize=8.5in,11in}
\setlength{\pdfpageheight}{\paperheight}
\setlength{\pdfpagewidth}{\paperwidth}


\title{SMT Solving}
\maketitle

\section{Introduction}

\begin {enumerate}

	\item What is an SMT solver?
	\begin{enumerate}
		\item Takes first order logic formulas, determines if they're Sat or Unsat
		\item Gives a model (solution) if a formula is Sat
	\end{enumerate}

	\item Uses cases
	\begin{enumerate}
		\item Program Verification
		\item Fuzz testing
		\item [MORE]
	\end{enumerate}
	

\end {enumerate}

\section{Group activity}
	The $n$-queens problem is to arrange $n$ queens on an $n$ by $n$ chessboard, so that none of the queens can attack each other.
	As queens can be moved any distance horizontally, vertically, or diagonally this is not easy to do by hand, especially for large values of $n$!

	By writing the problem as a first order logic formula, we can use an SMT solving to find a solution.  For small values of $n$, this is even possible to do by hand.

	You can install z3 by getting the source or binaries at:
	
	\url{https://github.com/Z3Prover/z3/wiki}
	
	or you can use it online at:
	
	\url{https://rise4fun.com/Z3}

[Demo] 

\section{Lab}
\begin {enumerate}

	\item For large values of $n$, writing the formula for $n$ by hand becomes tedious.
	In a language of your choice, write a program that takes $n$ as an argument,
	and generates the corresponding SMT formula.

	\item How many solutions there are for a given value of $n$?
	Modify your program to find all the solutions to the $n$ queens problem, rather than just one.
	This requires repeatedly calling the SMT solver, and forcing it to give you a new model from each call.
	To do this, negate all previous models, and add the negations to your formula. 

	[ADD EXAMPLE HERE]

	You'll know your done when the SMT solver returns Unsat.

	\item [REAL WORLD PROBLEM]

\end {enumerate}

\section{Wrap up}

\begin {enumerate}

\item

\end {enumerate}

\section {Links}

\begin{enumerate}
	\item z3 - Wiki
	
	\url{https://github.com/Z3Prover/z3/wiki}

	\item z3 - Rise4Fun (Online Tool)
	
	\url{https://rise4fun.com/Z3}

	\item SMT-LIB Homepage
	
	\url{http://smtlib.cs.uiowa.edu/index.shtml}
\end{enumerate}

\end{document}


